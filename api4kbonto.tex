% This is the main latex file for the
% submission to RuleML 2015 about
% API4KB ontologies
%
\documentclass[runningheads]{llncs}
\usepackage{amsmath,amssymb,amsfonts,textcomp}
\usepackage{url}
\usepackage{cite}
\usepackage{footnote}
\usepackage{hyperref}
\usepackage{enumitem}
%
\title{API4KB Ontologies: a Meta-API for Hybrid Knowledge Platforms}

\titlerunning{API4KB Ontologies}

\date{}
%
\begin{document}
%
\author{Davide Sottara\inst{1}\ \ Tara Athan\inst{2}}

\institute{
Arizona State University\\
\email{}
\and
Athan Services (athant.com), West Lafayette, Indiana, USA\\
\email{taraathan@gmail.com}
}
%
\maketitle

\begin{abstract}
API4KB ontologies target the basic administration
services as well as the retrieval and the modification of expressions in machine-readable knowledge representation and reasoning (KRR) languages within multi-language, multi-nature knowledge platforms.
KRR languages of concern include RDF, OWL, RuleML and Common Logic, and the knowledge platforms may support one or several of these.
The ontologies provide a metamodel, or abstract syntax, for the communications, including a classification of knowledge source by an abstraction hierarchy as well as the use of performatives (assert, query, ...), languages, logics, dialects and formats. Finally, a framework is provided for defining operations on knowledge sources and platforms.
\end{abstract}

%
\section{Introduction}
\label{intro}

\section{Upper Levels}
\label{upper}
The principle upper-level concepts in the API4KB ontology are
\begin{description}
\item[Application] architectural elements that may interact with (communicate with, call or be called by) other elements
\item[Channel] architectural element that forwards communications between Applications 
\item[KnowledgeSource] source of machine-readable information
\item[Operation] function (possibly with side-effects)
\item[Event] successful execution of an Operation request or call
\end{description}
These definitions are intentionally vague so as to be adaptable to a variety of paradigms. To allow for the greatest generality, we will not assume that communications are local (in the virtual address space), synchronous - although these properties could apply in some architectures. Communication channels may in general by many-to-many and uni- or -bidirectional, but a particular communication will have a unique sender. We will allow for failure, either in communication or in execution. Communication nodes (Applications) may be single-sorted or many-sorted, with sorts being characterized by the kind of communications that may be received and sent, and by the kind of sender, recipient or channel they may be received from or sent to.

A particular kind of KnowledgeSource is a Description, and each instance of the principle upper-level concepts has a Description, which may be expressed as an RDF graph.

\section{Description Schemas}
The properties that may be employed in Descriptions and the domains and ranges of these properties are specified in the following tables.



\section{Conclusion}
\label{conc}

%
% ---- Bibliography ----
%
\bibliographystyle{splncs}
\bibliography{api4kbonto}
\end{document}
